\chapter{LITERATURE REVIEW}

The papers in the literature review are used as a foundation to guide, the thesis, this included inspiration, the methodoly, and expected results.

\section{Video Image Recognition of Car Track Characteristics at Intersections \cite{paper_1}} 

The system relies on machine learning algorithms to interpret video data in real-time, helping to optimize traffic signal control, reduce congestion, and enhance safety. The study emphasizes its application in intelligent transportation systems and urban traffic flow optimization.

The use of YOLOv8, a popular deep learning model for object detection, because it offers a balance of speed and accuracy. YOLOv8 is efficient in processing real-time video streams, which is crucial for tracking vehicles at intersections. It excels in recognizing objects (like cars) in complex environments, and its architecture allows it to detect vehicles quickly while maintaining high precision.

\newpage
\section{A Lightweight Remote Sensing Small Target Image Detection Algorithm Based on Improved YOLOv8 \cite{paper_2}}
The authors optimize the YOLOv8 architecture by refining the feature  process and introducing techniques like multi-scale detection, anchor box adjustments, and a better feature fusion method. This helps the algorithm perform better with small targets.
ensuring more accurate detection without significantly increasing computational cost.

they conclude that, this makes it suitable for real-time applications in remote sensing, where detecting small objects like vehicles, ships, or buildings in satellite imagery is critical. The enhancements lead to better precision and recall.


\section{Traffic Detection and Enhancing Traffic Safety: YOLO V8 Framework and OCR for Violation Detection Using Deep Learning Techniques \cite{paper_3}}
The YOLOv8 algorithm is a deep learning model that uses a convolutional neural network (CNN) to detect and classify objects in input images. It uses bounding box regression and class prediction to identify objects and assign class probabilities. 

The model is pre-processed by converting the input video into frames, extracting features, and detecting objects. The YOLOv8 model then uses a grid to track objects, predicting bounding boxes and class probabilities. The model is fine-tuned on specific object classes and implemented for tracking, counting, and speed estimation.

This project consists of two main modules: one for vehicle count and speed detection, which uses a dataset of 800 images and video clips, and the other for number plate recognition. The first module uses computer vision techniques to accurately count vehicles and calculate speeds. The second module uses a Roboflow website dataset to extract license plate information, useful for applications like parking management and traffic monitoring. 

This project aims to create two modules with distinct functions in traffic monitoring. The number plate recognition module enhances functionality for recognizing and tracking vehicles based on their number plates, while the vehicle count and speed detection module provides insights into traffic flow and speed trends.

The study on traffic detection using the YOLOv8 framework and Optical Character Recognition (OCR) has shown promising results. The system accurately detects traffic violations in real-time, while OCR enhances its ability to identify and classify violations like illegal parking or speeding.

YOLOv8 and OCR provides a robust solution for automating violation detection processes, demonstrating the effectiveness of deep learning methodologies in enhancing traffic safety and enforcing traffic regulations.


\section{Smart Traffic Management System for Efficient Mobility and Emergency Response \cite{paper_4}}
The proposed solution combines AI with real-time data analytics to improve traffic light management. It incorporates real-time congestion response, algorithmic optimization, data-driven decision making, and feedback loop integration. 

The system uses a vast dataset of lane-specific traffic information to inform real-time adjustments. The Traffic Light Control System (OF) is activated, and the dataset undergoes preprocessing to ensure compatibility with the algorithms. A predictive model is trained using machine learning techniques, and a feedback loop is implemented to continuously evaluate the effectiveness of the decisions. 

The AI system also optimizes traffic flow during emergencies, adjusting timings to prioritize passage and ensure smooth traffic flow. This innovative approach aims to alleviate congestion and improve overall traffic flow, ultimately leading to sustainable and safer cities.

\newpage
\section{Utilizing Image Processing and the YOLOv3 Network for Real-Time Traffic Light Control \cite{paper_5}}
The paper presents the creation of a traffic light control system which is aimed at relieving congestion in urban areas. The system is designed based on the YOLOv3 object detection algorithm and is capable of counting both vehicles and pedestrians at intersections in real time using a Jetson Nano board for data processing. The research further shows that traffic lights can be controlled by taking into consideration factors such as the number of vehicles that have stopped, the number of vehicles that are crossing over and the number of pedestrians thereby enhancing the efficiency of the system which in turn reduces negative impacts on the environment.

The authors note the cost entailed in relief from congestion in UK being in the region of £37.7 billion per year and admonish that such inefficient delay is caused by the inadequate control of traffic lights that energy loss is also the outcome of that inefficient management of the traffic signal is responsible. The developed system loks set to achieve average precision with average vehicle counting for 95 percent as supported by this research.



\newpage
\section{Research on Vehicle Lane Change Warning Method Based on Deep Learning Image Processing \cite{paper_6}}
The paper presents the progress in work which is falling into the category of the development of the low-cost lane-changing warning system which works in the efforts to improve driving safety through monocular camera and deep learning algorithms. The main aim of the study is to detect when there is a pole vehicle-user behavior who would be changing the existing lane as this is paramount to mitigate risk of collision, consequently enhance safety of roads.

As a result, the authors enhanced the Mask Region-based Convolutional Neural Network (Mask R-CNN) aiming towards a vehicle target seeker. They used the K-means clustering technique to identify appropriate proportions of anchor frames for vehicle targets, which facilitated more precise candidate box generation without degrading the performance of the network. This led to introduction of a fresh anchor set and mAP increased by about 2.6 percent thanks to about 26 percent in the mean average precision improvement.

The system was trained in a vernacular fashion with a full with 66,389 vehicle targets which were adequately annotated for target annotation assuring their appropriate performance. The authors performed various evaluations and recorded an accuracy of 94.5 percent for detection of lane changing after identifying and validating marked images of lane changing sequences. This high accuracy demonstrates the effectiveness of the proposed system in real time applications.


\newpage
\section{Summary}
The reviewed studies highlight innovative uses of AI and deep learning for traffic management and safety. One study employs the YOLOv8 framework and Optical Character Recognition (OCR) to detect traffic violations in real-time, using modules for vehicle counting, speed estimation, and license plate recognition, achieving effective enforcement of traffic regulations. Another study integrates AI with real-time analytics to optimize traffic light control, dynamically responding to congestion and prioritizing emergency vehicles, thereby improving urban mobility and sustainability. Similarly, a traffic light control system based on YOLOv3 counts vehicles and pedestrians at intersections, achieving 95 percent accuracy in optimizing traffic flow and reducing urban congestion. Lastly, a lane-change warning system utilizing Mask R-CNN and K-means clustering achieves 94.5 percent accuracy in detecting lane changes, offering a cost-effective solution for collision risk mitigation and enhanced road safety. These approaches collectively demonstrate the transformative potential of AI in traffic management.







